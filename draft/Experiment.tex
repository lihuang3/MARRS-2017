\section{EXPERIMENT}\label{Experiment}

\subsection{Electromagnetic Platform}
We use a custom-made electromagnetic platform which consists of three orthogonal pairs of coils with separation distance equivalent to the outer diameter of a coil. 
The coils (18 AWG, Custom Coils, Inc) are powered by six SyRen10-25 motor drivers with Tekpower HY3020E DC power supply. 
An Arduino Mega 2560 provides PWM signal to control motor drives, and images are acquired using an IEEE 1394 camera (50 fps) with the region of interest approximate 20 mm$^2$. 
Each image has $379\times366$ pixels, and each pixel represents an area of 40 $\mu$m$^2$ of the workspace.  
We process microrobot detection and tracking in {\sc Matlab} using blob analysis and Kalman filters, and send control input ${\bf u}_t$ (i.e., the orientation of the magnetic field) to the Arduino Mega via USB serial port communication. 
In experiments, the electromagnetic platform (with iron cores) can provide over 300 Gauss magnetic fields along any direction in the 20 mm$^3$ workspace center.    

\subsection{Experiment Setup}
The vascular network we used to validate path planning and aggregation algorithms is shown in Fig. \ref{fig:coil} (b) and Fig. \ref{fig:maps} (b). This maze is made of two layers of acrylic cut using a Universal Laser Cutter, one layer as the base, and the other as the polygonal obstacles. %, with the layers joined by double-sided adhesive (MH-927 12-3, Adhesive Research Ireland Ltd). 
%The base layer is 5 mm thick, and the obstacle layer is 2 mm thick. 
The frame is a 20$\times$20 mm$^2$ square, and the channel width is 2 mm. 
In each experiment, the maze is filled with a mixture of microrobots and vegetable oil (0.45 mL) at the same concentration, and placed in the workspace center.
The microrobots are composed of ferromagnetic particles (30 microns Fe$_3$O$_4$, Alpha Chemicals). 
These microparticles aggregate into microrobots that  vary in sizes and shapes, with initial population over 300. 
Microrobots align with magnetic fields when the magnitude is larger than 100 Gauss. 
Because the density of ferromagnetic particles is over seven times larger than water, gradient fields provided by our electromagnetic platform are not able to drag microrobots around due to friction.
 Hence we create rotational fields to make the microrobots roll along the base.   
 Rolling a uniform field in the vertical plane at 5 Hz causes microrobots to move at an average velocity of 80 $\mu$m/s, and maximum velocity is over 350 $\mu$m/s.


\subsection{Validation of Aggregation Algorithms}
The results of the divide-and-conquer aggregation are compared with those of the benchmark heuristic aggregation as shown in Fig. \ref{fig:Exptcomp}. The running time is approximated by number of processed image frames for each experiment ($\approx$ 45 fps). The swarm population is estimated by averaging the number of pixels classified as robots in last 13500 frames ($\approx$ 5 min). With similar swarm populations, the average running time for the benchmark is 93,063 frames ($\approx 34.5$ min), and 60,628 frames ($\approx 22.5$ min) for the divide-and-conquer algorithm, which reduces by 34.9\%. Hence the divide-and-conquer aggregation outperforms the benchmark. 
      \begin{figure}[h]
      	\vspace{-3 mm}
      	\centering      	\begin{overpic}[width=0.7\columnwidth]{comp2}
      	\end{overpic}
      	
      	\caption{\label{fig:Exptcomp} Blue diamonds are benchmark data, and red circles are results for divide-and-conquer aggregation, with an experiment number next to each marker.} 
      \end{figure}
      
      