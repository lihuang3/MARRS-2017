%%%%%%%%%%%%%%%%%%%%%%%%%%%%%%%%%%%%%%%%%%%%%%%%%%%%%%%%%%%%%%%%%%%%%%%%%%%%%%%%
%2345678901234567890123456789012345678901234567890123456789012345678901234567890
%        1         2         3         4         5         6         7         8

\documentclass[letterpaper, 10 pt, conference]{ieeeconf}  % Comment this line out if you need a4paper

%\documentclass[a4paper, 10pt, conference]{ieeeconf}      % Use this line for a4 paper


                                                     % you want to use the \thanks command
\IEEEoverridecommandlockouts                              % This command is only needed if 
\overrideIEEEmargins                                      % Needed to meet printer requirements.

% See the \addtolength command later in the file to balance the column lengths
% on the last page of the document

% The following packages can be found on http:\\www.ctan.org
\let\proof\relax
\let\endproof\relax
\usepackage{amsthm}
\usepackage{graphics} % for pdf, bitmapped graphics files
\usepackage{epsfig} % for postscript graphics files
\usepackage{mathptmx} % assumes new font selection scheme installed
\usepackage{times} % assumes new font selection scheme installed
\usepackage[cmex10]{amsmath}
\usepackage{amssymb}  % assumes amsmath package installed
\usepackage{cite}
\usepackage{calc}
\usepackage{url}
\usepackage{stmaryrd}
\usepackage{algorithm}
\usepackage[noend]{algorithmic}
\usepackage{algorithmicext}
\usepackage{ifthen}
\newtheorem{fact}{Fact}
\newtheorem{theorem}{Theorem}
\newtheorem{lemma}{Lemma}
\newtheorem{claim}{Claim}
\newtheorem{remark}{Remark}
\newtheorem{definition}{Definition}
\newtheorem{corollary}{Corollary}
\newtheorem{proposition}{Proposition}
\usepackage{boldline}
\usepackage{makecell}
\usepackage{afterpage}
\usepackage{eucal}

\usepackage{cite}




\newcommand\blankpage{%
	\null
	\thispagestyle{empty}%
	\addtocounter{page}{-1}%
	\newpage}

\usepackage{hyperref}
\hypersetup{
	colorlinks =true,
	urlcolor = black,
	linkcolor = black
}
\usepackage[caption=false,font=footnotesize]{subfig}
\usepackage{overpic}
%\usepackage{subfigure}
%\usepackage{textcomp}
\graphicspath{{./pictures/}}
\usepackage[caption=false,font=footnotesize]{subfig}
\usepackage[usenames, dvipsnames]{color}
\usepackage{colortbl}
\newtheorem{prop}{Proposition}
\newtheorem{rmk}{Remark}


\title{\LARGE \bf 
%Microrobot Swarms Aggregation and Environment Interference with a Global Input 
%Aggregation of a Microrobot Swarm using a Global Input and Exploiting Environment Interactions
Path Planning and Aggregation for a Microrobot Swarm in Vascular Networks Using a Global Input 
}


\author{Li Huang and Aaron T. Becker% <-this % stops a space
\thanks{Li Huang and Aaron T. Becker are with the Department of Electrical and Computer Engineering, University of Houston, Houston, TX 77004, USA
        {\tt\small lhuang21@uh.edu}}%
%        Dayton, OH 45435, USA
%        {\tt\small b.d.researcher@ieee.org}}%
}


\begin{document}



\maketitle
\thispagestyle{empty}
\pagestyle{empty}

%%%%%%%%%%%%%%%%%%%%%%%%%%%%%%%%%%%%%%%%%%%%%%%%%%%%%%%%%%%%%%%%%%%%%%%%%%%%%%%%
\begin{abstract}
Microrobots have great potential for microassembly and non-invasive surgery applications. %to be used for  micro-electro-mechanical systems (MEMS) and medical applications, such as , etc.  
%We present efficient strategies for collecting a swarm of simple microrobots in 2D vascular networks using a global input.
 %These physically preliminary microrobots usually work in large populations, but they have limited capabilities of actuation, computation and communication. 
%{\color{red}\#}Potential applications of these microrobots require large populations, but individual microrobots have no actuation, computation, or communication capabilities.
% Therefore, a shared input signal (such as optical signal, gravitational force, or electro-magnetic fields) is employed to provide external actuation and navigation. 
Motivated by studies proposing MRI-guided drug delivery to tumor cells using magnetic micro carriers, this paper studies two major challenges of this problem: (i) microrobot swarm trajectory generation, and (ii) swarm aggregation using a global input. 
We propose an augmented RRT for trajectory generation to reduce environment interference, and a divide-and-conquer algorithm for swarm aggregation to improve performance.
%We propose Trajectory generation of a microrobot swarm is based on an obstacle-weighted RRT algorithm, {\color{red}\#}which gives a path optimized to guide the majority of micro particles from one topological region to another despite  environment interference.
 %an optimized path with the global control input to reduces environment interference. 
% The divide-and-conquer aggregation algorithm {\color{red}\#}presented in this paper employs the environment to drive all microrobots to a goal location with a shared input.
Simulations demonstrate the utility of these approaches in comparison to alternate heuristics. Our trajectory generation and aggregation strategies are implemented on a swarm of ferromagnetic microparticles in oil using a 6-coil electromagnetic system with image feedback. 
\end{abstract}

%\begin{figure}[h]
%	\centering
%	\begin{overpic}[width=0.9\columnwidth]{VasAgg}
%	\end{overpic}
%	
%	\caption{\label{fig:VasAgg} Microrobot swarm path planning and aggregation in a vascular network. The red circle marks the target region of aggregation. Obstacles are shown in blue polygons.}
%\end{figure}

%%%%%%%%%%%%%%%%%%%%%%%%%%%%%%%%%%%%%%%%%%%%%
\input{Intro}
%
%\section{RELATED WORK}\label{RelatedWork}

\section{PROBLEM FORMULATION}\label{ProbForm}

Consider that a bounded 2D space $G \subset \mathbb{R}^2$ is consisted of free space ($G_{free}$) and obstacle space ($G_{obs}$). $G_{free}$ is connected by paths of width at least $w$. A swarm of simple microrobots are initialized at random locations in $G_{free}$. These simple microrobots have on-board computation or communication, and they are guided by a global field. Their bodies are less than $w/10$, and mutual interaction is ignored. An $n$-microrobot swarm has $2n$ degree-of-freedom in a plane, and applying a global field only adds two constraints ($x$ and $y$ components) to the system. To break the symmetry of swarm motion with a global field, the environment is employed to assist swarm tasks. 

Our preliminary research explored path planning algorithms and aggregation strategies to collect these microrobots at a targeted location. We proposed an obstacle-weighted rapidly-exploring random tree (RRT) to plan near-medial-axis trajectories from any location to the goal, and benchmark heuristic and divide-and-conquer algorithms were developed for microrobot swarm aggregation in vascular networks. This paper presents an improved method for path planning with high efficiency and evaluate how well the environment can help with swarm aggregation. 

The microrobots are modeled as dots without area. At time $t$, the position of the $i$-th microrobot is denoted as $(x_t^i, y_t^i)$ and a global control input $u_t=(u_x, u_y)$ moves the microrobot to
\begin{equation}
\big(x_{t+1}^i, y_{t+1}^i\big) = \big(x_t^i, y_t^i\big)+\big(u_x,u_y\big), 
\end{equation}   
or the robot stops at the boundary of an obstacle.
\begin{figure}[h]
	\centering
	\begin{overpic}[width=1\columnwidth]{AggProc}
	\end{overpic}
	
	\caption{\label{fig:AggProc} Blue polygons represent obstacles, and white channels are free space. We place a red dot at the goal location. The dashed arc represents the region that corresponds to $F(n,t)<\sigma$. (a) Simulated aggregation process after 1 step, (b) 200 steps, (c) 500 steps, and (d) 800 steps.} 
\end{figure}



%
\section{Swarm Path Planning}\label{sec:path}

Our previous research built an obstacle-weighted rapidly-exploring random tree (RRT) planner to discover near-medial-axis routes to the goal location to deliver a microrobot swarm.  

Sampling-based motion planning algorithms have shown great success in exploring collision-free paths for many scenarios. Probabilistic roadmaps (PRMs) \cite{kavraki1998probabilistic} and rapidly-exploring random trees (RRTs) \cite{lavalle1998rapidly} are two popular planners. These planners generate random configurations in free space, connect them to create a graph of feasible paths, and link start and goal locations. In this paper, we focus on multi-shot 2D path planning with RRT and its extensions. Many attempts have shown success in improving the performance of RRTs near obstacles, such as narrow passage and tight region problems using a retraction strategy \cite{zhang2008efficient} \cite{tang2006obstacle}.  Typically, retraction-based planners bias a configuration towards a more desirable region, for example, sampling new configurations in narrow paths more densely, and growing the tree near medial axes of $G_{free}$. %The configuration generation of RRT usually behaves in  an unbiased manner, but with retraction, the resulting planner is steered towards the region of interest. 
As a result, less iterations are required to generate a feasible path connecting start and goal locations. 

%A problem arises when it comes to a microrobot swarm. As we consider simple microrobots under a global controller, moving one particular agent leads to all others drifting away from their original locations. 
Applying an RRT-based path to a microrobot swarm using a global input can lead to problems: moving one particular agent may cause all the others to drift away from their initial locations.
These locations may not be near any existing configurations on the tree ($\mathcal{T}$). Hence our RRT planner should have the following feature: for any robot $r_i$ in $G_{free}$, we have
\begin{equation}\label{Eqn:issue1}
\|{\bf p}^{r_i}-{\bf q}_v\|_2\le \epsilon,
\end{equation}
for some $\epsilon>0$,  where ${\bf q}_v$ is the nearest vertex in $\mathcal{T}$. We grow a tree in an unbiased manner, such that sampling configurations are distributed uniformly. We also require sufficient configurations to guarantee Eqn. \ref{Eqn:issue1}.
%
%an intuitive way of performing aggregation is moving each microrobot to the goal location iteratively until the cost function is less than certain threshold. This results a RRT planner of $O(N)$ pairs of paths, and each connects the goal ${\bf q}$ and the current location of microrobot $r_i (i=1,2,\dotsm, N)$. However, the microrobot system is dynamic since
\begin{figure}[h]
	\vspace{2 mm}
	\centering
	\begin{overpic}[width=0.7\columnwidth]{MethFig1}
	\end{overpic}
	
	\caption{\label{fig:MethFig1} Microrobot aggregation along different trajectories. 
		%The blue polygons are obstacles, and the white region is free space. 
		One swarm (green circles) follows a black solid trajectory near the medial axes, and they keep moving together. Another swarm, represented by red triangles, follows the shortest path to the goal, which traps some microrobots at the corner and slows the aggregation process}
\end{figure}

Another issue with microrobot swarms is the environment interference. For example, in Fig. \ref{fig:MethFig1} a gray dashed line (trajectory 1) is the shortest path to the goal location, and a black solid line (trajectory 2) represents a near-medial-axis path. Although trajectory 1 is shorter than 2, such a path significantly slows the aggregation process near obstacles. We propose an approach  which reroutes existing paths towards medial axes of free space. Compared to retraction-based planners, this approach replans a new route with existing configurations in $\mathcal{T}$ instead of generating biased sampling of tree nodes.   
\begin{table}\label{table1}
	
	\begin{tabular}{l}
		\Xhline{0.8pt}	 
		\\[-0.8em]
		\multicolumn{1}{c}{\bf {Variables, and functions used in Alg. \ref{alg:RRT} and \ref{alg:ARRT}}} \\
		\\[-1em]
		\hline
		\\[-0.8em]
		$V$ --- the set of vertices (configurations) of $\mathcal{T}$.\\
		\\[-0.5em]
		$V_{obs}$ --- the set of sampling nodes on the boundary of obstacles.\\
		\\[-0.5em]
		$V^*$ --- the set of configurations near medial axes of $G_{free}$. \\
		\\[-0.5em]
		$\pi({\bf q}_v)$ --- the predecessor of the configuration ${\bf q}_v$ in $\mathcal{T}$.\\
		\\[-0.5em]
		${\textit{Adj}}({\bf q}_v)$ --- the set of adjacent vertices. \\
		\hline
	\end{tabular}
\end{table}

    \begin{algorithm}[h]
    	
    	\begin{algorithmic}[1]
    		\STATE $ V = \{{\bf q}^g\},V_{obs}=\emptyset,\mathcal{T} = \{V,V_{obs}\}$
    		\WHILE{$|V|\le NumNode$}
    		\STATE ${\bf q}_{rand} \leftarrow$ a randomly generated point in $G$
    		\STATE ${\bf q}_{near} \leftarrow$ the nearest neighbor of ${\bf q}_{rand}$ in $V$
    		\STATE ${\bf q}_{new} \leftarrow$ extend ${\bf q}_{near}$ towards ${\bf q}_{rand}$ for unit length %+c\cdot\frac{{\bf q}_{rand}-{\bf q}_{near}}{\| {\bf q}_{rand}-{\bf q}_{near}\|_2}$
    		\IF{$({\bf q}_{near},{\bf q}_{new}) \cap G_{obs}=\emptyset$}
    		\STATE $V = V \cup \{{\bf q}_{new}\}$
    		%	\STATE $E = E \cup ({\bf q}_{near},{\bf q}_{new})$
    		\STATE $\pi({\bf q}_{new})={\bf q}_{near}$
    		\STATE $d\langle {\bf q}_{new}, {\bf q}^g\rangle = d\langle {\bf q}_{new}, {\bf q}_{near}\rangle + d\langle {\bf q}_{near}, {\bf q}^g\rangle   $
    		\ELSE
    		\STATE ${\bf q}_{obs} \leftarrow ({\bf q}_{near},{\bf q}_{new}) \cap \partial G_{obs}$
    		\STATE $V_{obs} = V_{obs} \cup \{{\bf q}_{obs}\}$
    		\ENDIF
    		\ENDWHILE
    		\STATE return $\mathcal{T}$
    	\end{algorithmic}
    	\caption{\textit {RRT}. Input: configuration space $G$, goal location ${\bf q}^g$, total number of configurations $NumNode$. Output: an RRT $\mathcal{T}$}
    	\label{alg:RRT}
    	
    \end{algorithm}
The basic RRT planner builds a connected tree rooted at the goal location, and samples tree nodes randomly in free space of $G$ to explore the graph. This process (Alg. \ref{alg:RRT}) proceeds as follows: to grow the tree, we generate a random point ${\bf q}_{rand}$ in $G_{free}$, and perform the nearest neighbor query (lines 3-4). Next, ${\bf q}_{near}$ is extended towards ${\bf q}_{rand}$ with unit length, and ends with ${\bf q}_{new}$. If the edge $({\bf q}_{near},{\bf q}_{new})$ is collision-free, which corresponds to a control input steering robots from ${\bf q}_{near}$ to ${\bf q}_{new}$, we add the new node to the tree and update the distance metric (lines 6-9).  Since all sampling points are connected to the tree, a robot can reach the goal location from any tree nodes simply by following their predecessors iteratively. 






\begin{algorithm}[h]
	\begin{algorithmic}[1]
		%		\STATE //Add obstacle weight to the tree nodes
		\STATE $  V^* = \{{\bf q}^g\}, \mathcal{T}_{obs} = \{V, V_{obs},V^*\}$
		\FOR{all ${\bf q}_v \in V$}
		\STATE ${\bf q}_{obs}^{near} \leftarrow$ the nearest neighbor of ${\bf q}_v$ in $V_{obs}$
		%\STATE $d\langle {\bf q}_v,\pi({\bf q}_v)\rangle = d\langle {\bf q}_v,\pi({\bf q}_v)\rangle+e^{-a\|{\bf q}_v-{\bf q}_{obs}^{near}\|_2+b}$
		\STATE $w({\bf q}_v) = e^{-a\|{\bf q}_v-{\bf q}_{obs}^{near}\|_2+b}$
		\IF{$w({\bf q}_v) < \zeta$}
		\STATE	$V^* = V^* \cup \{{\bf q}_v\}$
		\ENDIF 
		\ENDFOR
		\FOR{all ${\bf q}^*_{v} \in V^*$}
		\STATE $d\langle {\bf q}^*_v, {\bf q}^g\rangle=\infty$
		\FOR{all ${\bf q}^*_{u} \in \textit{Adj}({\bf q}^*_{v}) \cap V^* $  }
		\IF{$d\langle {\bf q}^*_v,{\bf q}^g\rangle > d\langle {\bf q}^*_v,{\bf q}^*_{u}\rangle+ w({\bf q}^*_u)+d\langle {\bf q}^*_{u},{\bf q}^g\rangle $}
		\STATE $\pi({\bf q}^*_v)={\bf q}^*_{u}$
		\STATE $d\langle {\bf q}^*_v,{\bf q}^g\rangle = d\langle {\bf q}^*_v,{\bf q}^*_{u}\rangle+ w({\bf q}^*_u)+d\langle {\bf q}^*_{u},{\bf q}^g\rangle $
		\ENDIF
		\ENDFOR
		\ENDFOR
		%		\STATE // Rerouting RRT based on obstacle weight
		\FOR{all ${\bf q}_v \in V$}
		\IF{$\pi({\bf q}_v) \notin V^*$}
		\STATE ${\bf q}^*_{near}\leftarrow$ the nearest neighbor of  ${\bf q}_v$ in $V^*$
		\STATE $\pi({\bf q}_v)={\bf q}^*_{near}$
		\ENDIF
		\ENDFOR
		\STATE return $\mathcal{T}_{obs}$
	\end{algorithmic}
	\caption{\textit {Obstacle-weighted RRT}. Input: the RRT $\mathcal{T}$ of Alg. \ref{alg:RRT}. Output: an obstacle-weighted RRT: $\mathcal{T}_{obs}$ }
	\label{alg:ARRT}
\end{algorithm}

Note that in Alg. \ref{alg:RRT}, lines 10-12 are different from the original RRT \cite{lavalle1998rapidly}: if there is a collision along the path, we retract ${\bf q}_{new}$ to the boundary of the obstacle ${\bf q}_{obs}$. ${\bf q}_{obs}$ is not considered as a valid configuration in $\mathcal{T}$, instead, we add it to $V_{obs}$ to grow an obstacle-weighted RRT. These obstacle nodes assist in  steering paths away from obstacles. 


\begin{figure}[h]
	\vspace{1.5 mm}
	\centering
	\begin{overpic}[width=0.8\columnwidth]{ARRT}
	\end{overpic}
	
	\caption{\label{fig:ARRT} (a) RRT (original): yellow dots are configurations of $\mathcal{T}$, and red dots are abandoned extensions within the blue obstacle. Trajectory generation relies on the shortest paths to the goal. (b) Obstacle-weighted RRT: green dots are near-medial-axis configurations $\in V^*$, yellow dots in the shadow are elements in $V$ affected by obstacles, and red dots inside the obstacle are retracted to the boundary (black dots) and added into $V_{obs}$. New paths tend to avoid near-obstacle regions, and approach near-medial-axis space.}
\end{figure}



We illustrate this process in Alg. \ref{alg:ARRT}, and compare it with the original RRT in Fig. \ref{fig:ARRT}. The weight of a node ${\bf q}_v$ in $\mathcal{T}$ is calculated as follows,
\begin{equation}\label{ObsDist}
	w({\bf q}_v)=e^{-a\|{\bf q}_v-{\bf q}_{obs}^{near}\|_2+b},
\end{equation}
where $a, b \in \mathbb{R}^+$. Therefore, weight decreases with distance from nearby obstacles. Hence, if we identify a gradient descent path to the goal with minimum-weight nodes, the new path tends to proceed near medial axes of free space.   
A near-medial-axis set of configurations is constructed as: 
\begin{equation}
V^*=\{{\bf q}_v\in V|w({\bf q}_v)<\zeta \},
\end{equation}
for some  $\zeta\in \mathbb{R}^+$. The trajectory generation (lines 7-16) is shown in Fig. \ref{fig:ARRT}(b). We trim the tree to remove edges not connecting to vertices in $V^*$, and perform adjacent neighbors query to regrow the tree towards near-medial-axis regions. Section \ref{sec:agg} shows that obstacle-weighted RRT decreases aggregation time. 

   











%
\section{Swarm Aggregation}\label{sec:agg}
This section presents a divide-and-conquer aggregation method with heuristic strategies to improve performance. 
The motivation behind microrobot swarm aggregation is efficient control strategies for drug delivery in vascular networks. However, a global input with a highly under-actuated swarm system makes it difficult constructing an optimal controller. Pioneering research has proposed different strategies for the aggregation/gathering problem, but most are element-wise algorithms, that is, performing the task in terms of individuals. For example, \cite{mahadev2016collecting} combines two agents each time and tests the efficiency of four heuristics. Our goal is to propose a swarm-level strategy to carry out swarm aggregation, and reduce time complexity compared to element-wise methods.


\subsection{Heuristic Aggregation}
A benchmark heuristic for aggregation is to move one microrobot to the goal, and then move the next agent. 
Repeat this till all robots gather near the goal location. 
In this paper, the benchmark heuristic moves the farthest microrobot to the goal.

We present the heuristic aggregation in Alg. \ref{alg:Heuristics}.
%, and the algorithm convergence is shown in Proposition \ref{prop:conv}. 
%Alg. \ref{alg:Heuristics} is correct 
with the following assumptions: (i) the graph $G$ is connected and bounded, and (ii) the goal location is inside a closed region (Def. \ref{def:closeregion}) at a dead end (Def. \ref{def:deadend}).
 The second assumption is inspired by the concept of discrete system transitions in \cite{bobadilla2011controlling}, where gates are constructed to guide transitions from one region to another. 
  In practice, the closed region indicates that once microrobots reach this area, it is hard for them to escape, given global inputs driving robots to ${\bf q}^g$. 
  This is reasonable and essential, because an aggregated swarm may disperse in an open region given a global control signal. In fact, chemotherapy molecules are designed to release from carriers once they reach the region of tumor cells \cite{tiwari2012drug}.    

\begin{algorithm}[h]
	\begin{algorithmic}[1]
		\WHILE {$F(n,t) > \sigma$} %  \frac{1}{n}\sum_{i=1}^{n}d\langle {\bf p}^{r_i}_t,{\bf q}^g\rangle
		\STATE $r_i \leftarrow $ the farthest robot
		\STATE ${\bf q}^{r_i}_v \leftarrow $ the nearest vertex $\in V$ to $r_i$ 	
		\STATE ${\bf u}_t \leftarrow$ move  $r_i$ towards ${\bf q}^g$ via ${\bf q}^{r_i}_v $
		\ENDWHILE
	\end{algorithmic}
	\caption{\textit {Heuristic Aggregation}. Input: $\mathcal{T}_{obs} = \{V, V_{obs}, V^*\}$,  initial positions $\{ {\bf p}^{r_i}_{t_0}\}$ of all robots $r_i$. Output: ${\bf u}_t$   }
	\label{alg:Heuristics}
\end{algorithm}



\begin{definition}\label{def:closeregion}
	{\normalfont (closed region.)} Considering a global input ${\bf u}_t$ that drives all robots to the goal ${\bf q}^g$, a closed region is a positive invariant set $\mathcal{M} \subset G_{free}$, ${\bf q}^g \in \mathcal{M}$.  Given ${\bf u}_t$ and a robot $r_i$, if ${\bf p}^{r_i}_{t_0} \in \mathcal{M}$ at $t_0$, then ${\bf p}^{r_i}_t\in \mathcal{M}$ for all $t>t_0$. $\mathcal{M}$ is bounded, so $\forall\; {\bf p}^{r_i} \in \mathcal{M}, \exists\;c>0$, such that
	\begin{equation}
	d\langle {\bf p}^{r_i}, {\bf q}^g \rangle< {c\sigma}, 
	\end{equation}

\end{definition}

\begin{definition}\label{def:deadend}
	{\normalfont (dead end.)} A dead end is a set $\mathcal{D}\subset \mathcal{M}, {\bf q}^g \in \mathcal{D}$. $\mathcal{D}$ has the following properties: 
	\begin{enumerate}
		\item $\forall {\bf p}^{r_i} \in \mathcal{D}$, $d\langle {\bf p}^{r_i}, {\bf q}^g \rangle< {\sigma}$;
		\item given that all robots \{$r_i$\} $\in G_{free}$ aggregate inside $\mathcal{M}$ and a global input ${\bf u}_t$ moves robots towards $\mathcal{D}$, if ${\bf p}^{r_i} \in \mathcal{D}$ at $t_0$, then ${\bf p}^{r_i}\in \mathcal{D}$ for all $t>t_0$.   
	\end{enumerate}
\end{definition}

%\begin{prop}\label{prop:conv}
%	{\normalfont (convergence.)} Given $n$ microrobots $\{r_i\}$, a map $G$ of size $m$, and a global input, {\normalfont Alg. \ref{alg:Heuristics}} converges to $\mathcal{D}$ as $t\rightarrow \infty$, that is, $\forall\; t>t_f(n, m)$, $F(n,t) < \sigma$, for some small value $\sigma >0$.      
%\end{prop}
%\begin{proof}
%	%Each time we consider the farthest robot, say $r_i$, move $r_i$ to the goal location ${\bf q}^g$, and then consider the next farthest robot. 
%	Since the goal location is inside a closed region, after $n$ iterations in Alg. \ref{alg:Heuristics}, we collect all robots in the bounded region $\mathcal{M}$,
%	\begin{equation}
%		F(n,t)=\frac{1}{n}\sum_{i=1}^{n}d\langle {\bf p}^{r_i}_t,{\bf q}^g \rangle< c\sigma.
%	\end{equation}
%	Then Alg. \ref{alg:Heuristics} drives robots from $\mathcal{M}$ to $\mathcal{D}$ with at most $n$ iterations,
%	\begin{equation}
%				F(n,t)=\frac{1}{n}\sum_{i=1}^{n}d\langle {\bf p}^{r_i}_t,{\bf q}^g \rangle< \sigma.
%	\end{equation}  
%	Hence, the robot swarm aggregation can be finished in finite time---$O(n)$ iterations, and $G$ is bounded, i.e., Alg. \ref{alg:Heuristics} converges to $\mathcal{D}$ within $t_f(n,m)$ time.
%\end{proof}
\subsection{Divide-and-Conquer Aggregation} 
This method recursively aggregates microrobots into a smaller region that contains the goal. 
This transformation depends on how we define ``a smaller region". 
 A proper definition of ``region" reduces aggregation time.   
Interestingly, if we drive each microrobot all the way to the goal location, the algorithm is transformed into the heuristic aggregation.

The divide-and-conquer technique has two stages. 
We begin by splitting the aggregation problem into subproblems in smaller regions. 
Then we recursively perform discrete region transitions of microrobot swarms. 

The first stage ``divide" performs map segmentation of vascular systems like Fig. \ref{fig:maps}. 
In these maps, vessels are connected by junctions, and most of them are T-junctions. 

\begin{definition}
	{\normalfont (Region $R_i$.)} We define a partition of a map $G_{free}$ as non-overlapping regions $\{R_i\}_{i = 1,2,\dotsm, N_R}$, such that $\{ R_i\in G_{free}|\bigcup\limits_{i=1}^{N_R}R_i = G_{free}, R_i\cap R_j = \emptyset, \forall i\neq j \}$.
\end{definition}

\begin{definition}\label{JTnode}
	{\normalfont (Vessel and junction.)}  Let $Q = \{{\bf q}^*_v  \cup \textit{adj}({\bf q}^*_v )\}$, where  ${\bf q}^*_v \in V^*$ and $\textit{adj}({\bf q}^*_v)$ are adjacent neighbors of ${\bf q}^*_v$ within a range. Fitting an ellipse to elements $\in Q$ in the $X$-$Y$ plane, if the eccentricity $<\delta_e$, for some $\delta_e>0$, then ${\bf q}^*_v$ is a junction node; otherwise, ${\bf q}^*_v$ is a vessel node.
\end{definition}

As shown in Fig. \ref{fig:StdMap1234}(a), we can separate junction nodes (green dots) from other nodes in straight vessels (orange dots) by their spatial distributions as specified in Def.\ \ref{JTnode}. Generally speaking, a junction is the zone where different branches are joined, so the fitting ellipse of a junction node and its adjacent neighbors is similar to a circle. 
These junction nodes can determine boundaries between regions. 
In our implementation, we take the maximal width of local channels as the range of adjacent neighbors.  
With these definitions, we perform map segmentation using results from obstacle-weighted RRT. 
This process is presented in Alg.\ \ref{alg:MapSeg}, and illustrated in Fig. \ref{fig:StdMap1234}:
\begin{enumerate}
	\item use near-medial-axis configurations ${\bf q}^*_v \in V^*$ to identify junction nodes (lines 1-4);
	\item partition the set of junction nodes ($V_J$) using Euclidean distance (line 5), and yield $N_R$ junction clusters;
	\item split free space into $N_R$ regions corresponding to the $N_R$ junction clusters (line 6);
	\item partition each region into branches $\{B_{j,k}\}_{k=1,2, \dotsm, N_{B_j}}$  by their orientations (lines 7-10), where $B_{j,k}$ is the $k$-th of $N_{B_j}$ branches in $R_j$, $N_{B_j}\le 3$. 
\end{enumerate}


\begin{table}\label{table2}
	\vspace{1\baselineskip}
	\begin{tabular}{l}
		\Xhline{0.8pt}	 
		\\[-0.8em]
		\multicolumn{1}{c}{\bf {Variables and functions used in Alg. \ref{alg:MapSeg} }}  \\
		\\[-1.1em]
		\hline
		%	Country Name     or Area Name& ISO ALPHA 2\\
		%	\hline
		\\[-0.8em]
		%		$V_J$ --- the set of junction vertices of $\mathcal{T}$.\\
		%		\\[-0.5em]
		$Clustering(V_J,$ `distance') --- partition elements of $V_J$ into junction\\ clusters with the Euclidean distance metric, and returns the \\ centroid of each junction $\{ {\bf q}^d_j \}$.\\
		\\[-0.5em]
		$RegionSeg\big(V^*,\{{\bf q}^d_{j}\}\big)$ --- partition elements of $V^*$ into clusters by \\junctions, and returns the region ID: $\psi({\cdot})$. First,  assign a unique \\region ID to the centroid of each junction, $\psi({\bf q}^d_{j})=R_j$. Next,\\ $\psi\big(\textit{Adj}({\bf q}^d_{j}) \big)=R_j$. Then, assign all descendants of $\textit{Adj}({\bf q}^d_{j})$ the \\same region ID. \\
		\\[-0.5em]
		$Clustering(S,{\bf q}^d_{j}$ `orientation') --- partition elements $s_i\in S$ into\\ clusters by the orientation of a directed edge $(s_i,{\bf q}^d_{j})$, and returns \\ $\{ {\bf q}^o_{j,k} \}$ the mean orientation of each cluster. \\
		\\[-0.5em]
		$BranchSeg\big(S,\{{\bf q}^{o}_{j,k}\}\big)$--- assign a branch ID $\phi(\cdot)$ to each element \\of $S$ by orientation, where ${\bf q}^o_{j,k}$ is the orientation of branch $B_{j,k}$.\\
		\hline
	\end{tabular}
\end{table}

\begin{algorithm}[h]
	\begin{algorithmic}[1]
		\STATE $V_{J}=\{{\bf q}^g\}$
		\FOR{all ${\bf q}^*_{v} \subset V^*$ }
		\IF {${\bf q}^*_v$ is a junction node}
		\STATE $V_J = V_J\cup \{{\bf q}^*_{v}\} $
		\ENDIF
		\ENDFOR	
		\STATE $\{{\bf q}^{d}_{j}\}_{j=1,2,\dotsm, N_d}\leftarrow Clustering(V_J,$`distance') 
		\STATE $\{\psi({\bf q}^*_v)=R_j\}_{j=1,2,\dotsm, N_R}\leftarrow RegionSeg(V^*, \{{\bf q}^{d}_{j}\}$)
		\FOR{$(j=1;j=j+1;j\le N_R)$}
		\STATE $S \leftarrow \{{\bf q}^*_v\in V^*|\psi({\bf q}^*_v) =R_j\}$
		\STATE $\{{\bf q}^{o}_{j,k}\}_{k=1,2,\dotsm, N_{B_j} } \leftarrow Clustering(S, {\bf q}^{d}_{j}$ `orientation') 
		\STATE $\{\phi({\bf q}^*_v)=B_{j,k}\}_{k=1,2,\dotsm, N_{B_j}}\leftarrow BranchSeg(S, \{{\bf q}^{o}_{j,k}\}$) 
		\ENDFOR
		
	\end{algorithmic}
	\caption{\textit {Map Segmentation}. Input: The obstacle-weighted RRT $\mathcal{T}_{obs} = \{V, V_{obs}, V^*\}$. Output: Map segmentation: $M = \{V_J, \psi(V^*), \phi(V^*)\}$  }
	\label{alg:MapSeg}
\end{algorithm}
% Step 1:  Then we partition the set of junction nodes ($V_J$) using Euclidean distance as the distance metric and it yields $N_d$ clusters represented by their centroid coordinates (line 5). Next, the map can be split into $N_d$ regions corresponding to the $N_d$ clusters of junctions nodes (line 6). Finally, each region is further partitioned into branches by their orientations (lines 7-10).

%We describe some issues of Alg. \ref*{alg:MapSeg} implementation here. 
In step 3, region segmentation proceeds in three phases. 
First we select a cluster of $V_J$ and set the junction nodes as seeds. 
Then we grow the region with these seeds by adding their descendants  generation by generation in $\mathcal{T}_{obs}$. 
The region expansion stops at the next junction. In step 4, branch segmentation is a result of clustering. 
Considering all ${\bf q}^*_v$ in the $j$-th region, we connect ${\bf q}^d_{j}$ to ${\bf q}^*_v$, where ${\bf q}^d_{j}$ is the centroid of the $j$-th junction cluster. 
Branches can be obtained by clustering all these directed edges (${\bf q}^d_{j}, {\bf q}^*_v$).              
    
\begin{figure}[h]
	\centering
	\begin{overpic}[width=0.9\columnwidth]{StdMap1234}
	\end{overpic}
	
	\caption{\label{fig:StdMap1234} Map segmentation illustration for Alg. \ref{alg:MapSeg}. The goal (red dot) is located at (99,5).  (a) represents step 1 and 2, where yellow points are nodes in $V^*$, and green points are junction nodes in $V_J$. (b) and (c) show step 3, where regions are marked as different colors. (d) illustrates step 4 in a region, where branches are marked as different colors. }
\end{figure}

\begin{definition}\label{def:robust}
	{\normalfont (aggregation robustness.)} Considering a multi-robot system with a global input, we define robustness as the ability of an aggregation procedure to keep aggregated swarms in their regions despite influences of a global controller.
\end{definition}

If microrobots keep escaping from the goal region, or the procedure gets trapped in a local minimum (e.g. moving a swarm of microrobots back and forth without progress), we say it is not robust. 
The map segmentation is essential for identifying relatively closed regions $\{R_i\}$ which we use to perform discrete region transitions on swarms.  
 

With map segmentation, we are able to process the second stage ``conquer". 
This process is presented in Alg. \ref{alg:Agg} and illustrated in Fig. \ref{fig:DCproc}. The assumptions of Alg. \ref{alg:Heuristics} hold. 
A global planner moves a swarm of microrobots from region $R_j$ to $R_{j.next}$, where region $R_j$ and $R_{j.next}$ share an edge, and region $R_{j.next}$ is closer to the goal, $d \langle {\bf q}^d_{j.next}, {\bf q}^g\rangle < d \langle {\bf q}^d_j, {\bf q}^g\rangle$. 
A local planner assigns priorities to microrobots at different branches of region $R_j$, and leads them to the closer region $R_{j.next}$.              

\begin{figure}[h]
	\vspace{1.8 mm}
	\centering
	\begin{overpic}[width=0.8\columnwidth]{DCproc1} % timecom
	\end{overpic}
	
	\caption{\label{fig:DCproc} Black circles are microrobots, and regions are marked by colored outline. (a) Robots exist in the purple region $R_i$ and the orange region $R_j$. $R_j$ is the farthest region from the goal (red dot). (b) and (c) The branch filled with orange has higher priority. $R_{j.next}$ is marked as green. A control input drives the robot $r_{k_m,i_m}$ (hollow circle) to $R_{j.next}$. (d) Complete a discrete region transition for a swarm from $R_j$ to $R_{j.next}$.  }
\end{figure}


This divide-and-conquer algorithm consists of three \textit{while} loops: 
(i) identify the farthest region $R_{j_m}$ that microrobots exist  (Fig. \ref{fig:DCproc}(a)), and $d \langle {\bf q}^d_j, {\bf q}^g\rangle$ is the cost from the $j$-th junction centroid to goal; 
(ii) pick a branch with the highest priority, i.e., with more robots closer to the junction centroid ${\bf q}^d_{j_m}$ (Fig. \ref{fig:DCproc}(b)), computed by 
\begin{equation}
\upsilon(B_{j_m,k}) = \sum_{i=1}^{\#\text{robots in } B_{j_m,k}}\gamma^{s_d(i)},   
\end{equation} 
with $0<\gamma<1, s_d(i)=d\langle {\bf p}^{r^{k_m,i}}, {\bf q}^d_{j_m}\rangle,  r^{k_m,i}$ the $i$-th robot inside branch $B_{j_m,k_m}$; 
(iii) identify the closest robot $r_{k_m,i_m}$ to ${\bf q}^d_{j_m}$, and drive it to the nearest ${\bf q}^r_v \in V$, and then move it towards some ${\bf q}_{R_{jm.next}}\in V$ in $R_{jm.next}$  (Fig. \ref{fig:DCproc}(b) and (c)). After moving all robots in $R_{j_m}$ to $R_{j_m.next}$, we complete a discrete region transition for a swarm  (Fig. \ref{fig:DCproc}(d)). 





\begin{algorithm}[h]
	\begin{algorithmic}[1]
		\WHILE {$F(n,t) > \sigma$} %  \frac{1}{n}\sum_{i=1}^{n}d\langle {\bf p}^{r_i}_t,{\bf q}^g\rangle
		\STATE $R_{j_m} \leftarrow \textrm{argmax}\; d \langle {\bf q}^d_j, {\bf q}^g\rangle, j = 1,2, \dotsm, N_R$ 
		\WHILE {there exists any robot in $R_{j_m}$}
		\STATE $B_{j_m,k_m} \leftarrow \underset{B_{j_m,k}\in R_{j_m}}{\textrm{argmax}}\; \upsilon(B_{j_m,k}), k=1,2,\dotsm, N_{B_j}$
		\WHILE{there exist any robot in $B_{j_m,k_m}$}
		\STATE $r_{k_m,i_m} \leftarrow \textrm{argmin}\; d\langle {\bf p}^{r_{k_m,i}}_{t},{\bf q}^d_{j_m} \rangle $, $ \forall r_{k_m,i}$ in $B_{j_m,k_m}$ 
		\STATE ${\bf q}^r_v \leftarrow$ the nearest vertex $\in V$ to $r_{km,im}$ 	
		\STATE ${\bf u}_t \leftarrow$ move  $r_{km,im}$ towards ${\bf q}_{R_{jm.next}}$ via ${\bf q}^{r}_v$
		\ENDWHILE
		\ENDWHILE
		\ENDWHILE
%		\STATE return ${\bf u}$
	\end{algorithmic}
	\caption{\textit {Divide-and-conquer Aggregation}. Input: $\mathcal{T}_{obs} = \{V, V_{obs}, V^*\}$, $M = \{V_J, \psi(V^*), \phi(V^*)\}$,  initial positions $\{ {\bf p}^{r_i}_{t_0}\}$ of all robots $\{r_i\}$. Output: ${\bf u}_t$   }
	\label{alg:Agg}
\end{algorithm}
 

%
%\begin{prop}\label{prop:reverse}
%	For a local map $\mathcal{G} = R_j \cup R_{j.next}$, considering a trajectory generated by RRT/obstacle-weighted RRT, the aggregation from region $R_j$ to $R_{j.next}$ is robust.   
%\end{prop}
%\begin{proof}
%	 The trajectories in $\mathcal{G}$ are part of the global map $G$, heading to ${\bf q}^g$. The aggregation assignment is to navigate robots from $R_j$ to $R_{j.next}$. Considering a global input that steers robots to ${\bf q}^g$, we say an aggregation is robust in $\mathcal{G}$ if it does not result in reentry of robots to any branches. 
%	
%	The proof proceeds by contradiction.
%	A trajectory connecting $R_j$ and $R_{j.next}$ has three steps: navigating from $R_j$ to the junction, getting through the junction, and heading to the next junction in $R_{j.next}$. Let $\theta_i$ ($ i = 1,2,3$) denotes the approximate orientation of these steps, respectively. And all these trajectories share the same junction. Assuming that there exists a trajectory with its control input that leads to reentry of some robots to $B_{j,k}$, where $B_{j,k}$ is the $k$-th branch of $R_j$, this trajectory should reverse the steps of some planned trajectory ($\theta^a_i,i = 1,2, 3$), that is, in step 1, the trajectory takes the orientation $(\theta^a_3+\pi)$ to get to the junction, and then navigates through the junction heading towards $(\theta^a_2+\pi)$. However, the orientation from the junction to $R_{j.next}$ is restricted to $\theta^a_2$ with small deviation because all trajectories from $R_j$ to $R_{}j.next$ share this junction. This reaches a contradiction. Hence the aggregation in $\mathcal{G}$ is robust.               	 
%\end{proof}
%\begin{prop}\label{prop: relativeclosedreg}
%	Given the global input that steers robots to ${\bf q}^g$, $R_{j.next}$ is a relatively closed region in $\mathcal{G}$, where $\mathcal{G}=R_j \cup R_{j.next}$. 
%\end{prop}
%
%\begin{proof}
%``Relatively" refers to with high probability. From Proposition \ref{prop:reverse}, we can easily conclude that $R_{j.next}$ is a relatively closed region in $\mathcal{G}$.
%\end{proof}


%\begin{prop}
%	{\normalfont (convergence.)} Given a swarm of microrobots $\{r_i\}$, the map $G_{free}$ of size $m$, the map segmentation $\{R_i\}$, and the global input, {\normalfont Alg. \ref{alg:Agg}} converges to $\mathcal{D}$ as $t\rightarrow \infty$, that is, $\forall; t>t_f(N_R,m)$, $F(n,t) < \sigma$, for some small value $\sigma >0$.  
%\end{prop}
%
%\begin{proof}
%	%Each time we consider the farthest robot, say $r_i$, move $r_i$ to the goal location ${\bf q}^g$, and then consider the next farthest robot. 
%	Considering a region $R_j$, with $R_{j.next}$ and $\mathcal{G}=R_j\cup R_{j.next}$, the third \textit{while} loop (lines 5 - 8 in Alg. \ref{alg:Agg}) can moves all robots from each branch of $R_j$ to $R_{j.next}$ within 3 iterations, because a region has at most 3 branches in practice and $R_{j.next}$ is a relatively closed region in $\mathcal{G}$ (Proposition \ref{prop: relativeclosedreg}). This process can repeat at most $O(N_R)$ times since there are $N_R$ regions in total. Note that line 2 and line 4 take constant time. Hence Alg. \ref{alg:Agg} drives robots from all over the map to $\mathcal{D}$ in $O(N_R)$ iterations, i.e., Alg. \ref{alg:Agg} converges to $\mathcal{D}$ within $t_f(N_R, m)$ time.
%\end{proof}


To analyze time complexity of the divide-and-conquer recurrence, we need the following assumptions:
(i) the map is connected and bounded, 
(ii) ``closed region" and ``dead end" definitions, 
and (iii) aggregation time is proportional to map area and population. 
Let $G$ denote a map with $Area(G) = m,\; Population(G) = n,\; Density(G) = \rho=n/m$.
If $T(mn)$ is the running time for map $G$, we start from region $R_{j_m}$, $G^{\prime}(0) = G-R_{j_m}$, $Area(G^{\prime}(0))=\xi m, \;Population(G^{\prime})=\xi\rho m$, where $\xi$ is a discount factor. 
Level 0 of recurrence is:
\begin{equation}\label{DC} 
T(mn) = T(\xi mn)+f\big((1-\xi)\rho m\cdot (1-\xi)m\big),
\end{equation}
where $f((1-\xi)^2\rho m^2)$ denotes the aggregation time in $R_{j_m}$, with $(1-\xi)\rho m$ the population and $(1-\xi)m$ the area. After we move out all robots in $R_{j_m}$, the aggregation map shrinks from $G$ to $G^{\prime}(0)$.

We can easily derive the recursive form for level $i$ using two models. 
In the first recurrence model, we assume that the aggregation map shrinks with a constant discount factor $\xi$ each time, then $Area(G^{\prime}(i))=\xi\cdot Area(G^{\prime}(i-1))=\xi^{i+1}m$,
 \begin{equation}\label{Eqn:DCj1} 
 T(\xi^imn) = T(\xi^{i+1}mn)+f\big(\xi^{i}(1-\xi)\rho m\cdot \xi^i(1-\xi)m\big),
 \end{equation}
where the density is assumed to be a constant in $f(\cdot)$. 
In fact, the density decreases with aggregation since microrobots overlap. 
So this assumption does not reduce the difficulty of the subproblem. 
The base case is $T(n) = f(n)$ with $m = 1$, and we simplify $f(x)$ with a linear model $f(x) = kx$, then 
\begin{equation}\label{Eqn:timecmp}
\begin{split}
	T(mn) = \sum_{i=0}^{\log_{1/\xi}m} f\big(\xi^{2i}(1-\xi)^2\rho m^2\big)\\=k\rho m^2(1-\xi)^2\sum_{i=0}^{\log_{1/\xi}m}\xi^{2i}.
\end{split}
\end{equation}
Assuming $\log_{1/\xi}m$ is an integer, and $m\gg\xi$,  Eqn. \ref{Eqn:timecmp} can be simplified to
\begin{equation}\label{Eqn:timecmpS}
\begin{split}
T(mn) =k\rho m^2\big(\frac{2}{1+\xi}-1\big),
\end{split}
\end{equation}

In the second model, we reduce the map by a constant area $(1-\xi)m$ each time, then level $i$ has the form
  \begin{equation}\label{Eqn:DCj2}
  \begin{split}
 T\big( (1-i(1-\xi))mn \big) =
 T\big( (1-(i+1)(1-\xi))mn \big)\\+f\big( (1-\xi)^2\rho m^2 \big).
  \end{split} 
  \end{equation}   
Hence, we have 
\begin{equation}\label{Eqn:DCj2S}
T(mn) = k\rho m^2\sum_{i=0}^{1/(1-\xi)}(1-\xi)^2
\end{equation}
Assuming $\frac{1}{1-\xi}$ is an integer, we can write Eqn. \ref{Eqn:DCj2S} as
\begin{equation}\label{Eqn:DCj2SS}
T(mn) = k\rho m^2(1-\xi)(2-\xi)
\end{equation}
 
 
The performance of different discount factors $\xi$ is shown in Fig. \ref{fig:timecom}. 
As $\xi$ increases, the scaled running time decreases fast, despite some fluctuations in the second model. 
This means that the more we reduce the map size each time, the less efficient divide-and-conquer aggregation becomes. 
As $\xi\rightarrow 0$, we actually have the heuristic aggregation instead. 
This is equivalent to decreasing the map size from $m$ to 1 ($\xi = \frac{1}{m}$) with one recurrence. 
For both models (Eqn. \ref{Eqn:timecmpS} and \ref{Eqn:DCj2SS}), as $\xi\rightarrow\frac{1}{m}$, $T(mn)\rightarrow O(m^2)$; as $\xi\rightarrow 1-\frac{k^*}{m}$, for some $k^*\in \mathbb{R}^+$, $k^*\ll m$, $T(mn)\rightarrow T(m)$. 
Hence, the divide-and-conquer strategy makes it possible to reduce time complexity from $T(m^2)$ to $T(m)$. Note that $k^*$ is dependent on junctions in a map: the finer we can split the map, the smaller $k^*$ is.     

\begin{figure}[h]
	\centering
	\begin{overpic}[width=0.8\columnwidth]{timecomp2} % timecom
	\end{overpic}
	
	\caption{\label{fig:timecom} Running time estimation of the first recurrence model in Eqn. \ref{Eqn:timecmp} and the second recurrence model in Eqn. \ref{Eqn:DCj2S} }
\end{figure}
\begin{figure}[h]

	\centering
	\begin{overpic}[width=1\columnwidth]{maps1}
	\end{overpic}
	
	\caption{\label{fig:maps} Blue polygons represent obstacles, and white channels are free space. We place a red dot at each goal location. These maps increase in size and complexity: (a) T-junction map, (b) a vascular network, and (c) a larger vascular network.} 
	\vspace{-5 mm}
\end{figure}
%\begin{itemize}
%\item the challenge 
%\item discrete step
%\item environment interference ratio  
%\item different from grid world
%\item priori knowledge about the environment is available
%\item Gaussian Sampling \cite{boor1999gaussian}
%\item in order to acquire and process the information required for palnning the motion of the obj, we propose the use of a network of cooperating cameras with a top view of area wher the obj can be moved. 
%\item local planner, global planner, merge local info to global plan. 
%\item We cope with the aggregation part by introducing a number of heuristic strategies.
%\item Since sensing errors are an intrinsic property of tye system, we consider them as internal parameters, and we evaluate how they influence the performance.
%\item The aim of this work is to show that our distributed palnning system can find near optimal paths, makes an efficient utilization of computatoion and communication resources, is robust to increasing sensing errors, and its performance scales with the size of the enbbironment and the number of cameras.++
%\item The aggregation method we propose consists of ? phases: (i) Region Processing (ii) Branch Processing  (iii) Local Planning
%\item map segmenting (k-means and image segmentation)
%\item obstacle skeleton
%\item Heuristics to reduce local minima attraction
%\item RRT is to identify a gradient descent trakectory in this field.
%\item under-actuated system
%\item Steering function (maybe steering heuristics?)
%\item Local planner: drive agents in the region of interest to the next level, while minimizing cost function.
%\item RRt construct a tree where root is the target location s. Each node $t$ in the tree corresponds to a collision-free path while each edge $(u,v) \in E$ belongs to a contol input   
%\item low Reynolds number motion
%\item 
%\end{itemize}


\section{SIMULATION}\label{Simul}
\begin{figure*}[h] % !htb
	\centering
	\begin{overpic}[width=0.67\columnwidth]{SimuPlot1}\end{overpic}
	\begin{overpic}[width=0.67\columnwidth]{SimuPlot2}\end{overpic}
	\begin{overpic}[width=0.67\columnwidth]{SimuPlot3}\end{overpic}	
	\caption{\label{fig:SimuPlot} Particle aggregation in maps (a,b,c). The violin plot shows the probability density of the simulation data and the black line indicates the mean value. We performed 30 simulations for each combination of methods and swarm populations.}
\end{figure*}
We report the simulation results to evaluate our path planning approaches, RRT and obstacle-weighted RRT (OWRRT), compare the divide-and-conquer aggregation (DCA) with the heuristic aggregation (HRA), and study the impact of map and swarm population. We perform three sets of simulations, and in each set, we present two algorithms for aggregation and two methods for path planning. 






Path planning and aggregation are carried out in three simulated maps (Fig. \ref{fig:maps}), including a T-junction map, and two vascular networks. 
The obstacles are marked as blue polygons, and free space is white. 
To initialize, we place $n$ microrobots randomly in free space, where $n \in \{2^1, 2^2, 2^3, \dotsm, 2^{10}\}$, and each microrobot is represented by a point with no area. 
Given a global input ${\bf u}_t$ at time $t$, all microrobots will move towards the assigned direction for one discrete step of unit length (Eqn. \ref{Eqn:dynam1} and \ref{Eqn:dynam2}). 
The goal is to gather microrobots to the goal location. 
In practice, we define the task is accomplished if the average position of the swarm is near the goal, or $F(n,t) < \sigma$ (dashed circle in Fig. \ref{fig:maps}). 
We count the total number of steps to approximate the running time for swarm aggregation in a map. 
In each map, ten different microrobot populations are used. 
For each population, we perform 30 simulations for three combinations of aggregation algorithms and path planning methods respectively: DCA+OWRRT, HRA+OWRRT, and HRA+RRT. 
The results of simulations are compared using violin plots in Fig. \ref{fig:SimuPlot}.

We evaluate the performance of these algorithms by their average running time (number of steps) and data distributions. 
DCA+OWRRT outperforms any other combinations in all these simulation when the swarm population is large enough ($n \ge 2^3$). 
The average aggregation time of DCA+OWRRT does not grow as fast as others, and it tends to approach an upper bound asymptotically in each vascular network. 
Also, this combination shows reliability and efficiency with different environments and swarm populations. 
For each independent trial, the aggregation time has small standard deviation. 
Neither HRA+OWRRT nor HRA+RRT can compete with DCA+OWRRT in average aggregation time when the swarm size is greater than $2^3$. 
The average running time of HRA+OWRRT and HRA+RRT increases with $\log n$ in most cases with large standard deviation, and the worst case can be extremely inefficient.         
%\begin{figure}[h]
%	\centering
%	\begin{overpic}[width=0.7\columnwidth]{Tmap}
%	\end{overpic}
%	
%	\caption{\label{fig:Tmap} T-map }
%\end{figure}



%\begin{figure}[h]
%	\centering
%	\begin{overpic}[width=0.9\columnwidth]{figVB}
%	\end{overpic}
%	
%	\caption{\label{fig:figVB} Vascular network 2}
%\end{figure}








%\begin{figure}[h]
%	\centering
%	\begin{overpic}[width=0.8\columnwidth]{SimuPlot1}
%	\end{overpic}
%	
%	\caption{\label{fig:SimuPlot1} Particle aggregation in map (a). The violin plot shows the probability density of the simulation data and the black line indicates the mean value. We performed 30 simulations for each combination of methods and swarm populations.}
%\end{figure}
%
%
%\begin{figure}[h]
%	\centering
%	\begin{overpic}[width=0.8\columnwidth]{SimuPlot2}
%	\end{overpic}
%	
%	\caption{\label{fig:SimuPlot2} Particle aggregation in map (b)}
%\end{figure}
%
%\begin{figure}[h]
%	\centering
%	\begin{overpic}[width=0.8\columnwidth]{SimuPlot3}
%	\end{overpic}
%	
%	\caption{\label{fig:SimuPlot3} Particle aggregation in map (c)}
%\end{figure}






\section{EXPERIMENT}\label{Experiment}

\subsection{Electromagnetic Platform}
We use a custom-made electromagnetic platform which consists of three orthogonal pairs of coils with separation distance equivalent to the outer diameter of a coil. 
The coils (18 AWG, Custom Coils, Inc) are powered by six SyRen10-25 motor drivers with Tekpower HY3020E DC power supply. 
An Arduino Mega 2560 provides PWM signal to control motor drives, and images are acquired using an IEEE 1394 camera (50 fps) with the region of interest approximate 20 mm$^2$. 
Each image has $379\times366$ pixels, and each pixel represents an area of 40 $\mu$m$^2$ of the workspace.  
We process microrobot detection and tracking in {\sc Matlab} using blob analysis and Kalman filters, and send control input ${\bf u}_t$ (i.e., the orientation of the magnetic field) to the Arduino Mega via USB serial port communication. 
In experiments, the electromagnetic platform (with iron cores) can provide over 300 Gauss magnetic fields along any direction in the 20 mm$^3$ workspace center.    

\subsection{Experiment Setup}
The vascular network we used to validate path planning and aggregation algorithms is shown in Fig. \ref{fig:coil} (b) and Fig. \ref{fig:maps} (b). This maze is made of two layers of acrylic cut using a Universal Laser Cutter, one layer as the base, and the other as the polygonal obstacles. %, with the layers joined by double-sided adhesive (MH-927 12-3, Adhesive Research Ireland Ltd). 
%The base layer is 5 mm thick, and the obstacle layer is 2 mm thick. 
The frame is a 20$\times$20 mm$^2$ square, and the channel width is 2 mm. 
In each experiment, the maze is filled with a mixture of microrobots and vegetable oil (0.45 mL) at the same concentration, and placed in the workspace center.
The microrobots are composed of ferromagnetic particles (30 microns Fe$_3$O$_4$, Alpha Chemicals). 
These microparticles aggregate into microrobots that  vary in sizes and shapes, with initial population over 300. 
Microrobots align with magnetic fields when the magnitude is larger than 100 Gauss. 
Because the density of ferromagnetic particles is over seven times larger than water, gradient fields provided by our electromagnetic platform are not able to drag microrobots around due to friction.
 Hence we create rotational fields to make the microrobots roll along the base.   
 Rolling a uniform field in the vertical plane at 5 Hz causes microrobots to move at an average velocity of 80 $\mu$m/s, and maximum velocity is over 350 $\mu$m/s.


\subsection{Validation of Aggregation Algorithms}
The results of the divide-and-conquer aggregation are compared with those of the benchmark heuristic aggregation as shown in Fig. \ref{fig:Exptcomp}. The running time is approximated by number of processed image frames for each experiment ($\approx$ 45 fps). The swarm population is estimated by averaging the number of pixels classified as robots in last 13500 frames ($\approx$ 5 min). With similar swarm populations, the average running time for the benchmark is 93,063 frames ($\approx 34.5$ min), and 60,628 frames ($\approx 22.5$ min) for the divide-and-conquer algorithm, which reduces by 34.9\%. Hence the divide-and-conquer aggregation outperforms the benchmark. 
      \begin{figure}[h]
      	\vspace{-3 mm}
      	\centering      	\begin{overpic}[width=0.7\columnwidth]{comp2}
      	\end{overpic}
      	
      	\caption{\label{fig:Exptcomp} Blue diamonds are benchmark data, and red circles are results for divide-and-conquer aggregation, with an experiment number next to each marker.} 
      \end{figure}
      
      
%%%%%%%%%%%%%%%%%%%%%%%%%%%%%%%%%%%%%%%%%%%%%
\section{CONCLUSIONS}
This paper compared two path-planning methods and two control strategies applied to the problem of aggregating microrobot swarms in vascular networks using a global input. 
Although RRT creates an obstacle-free path from initial locations to a goal for a single robot, this path is not ideal for swarms. 
We propose an obstacle-weighted RRT that steers microrobots towards near-medial-axis regions to reduce environment interference.   
A divide-and-conquer strategy is employed to perform swarm-level aggregation via discrete region transitions. 
Compared to the benchmark strategy, the divide-and-conquer aggregation reduces the task time complexity.    
%In future work , we plan to augment our research in theory and experiment. 
%We will prove the convergence of our aggregation algorithms, compare more path planning techniques, test microrobots in different maps and construct a more delicate electromagnetic platform for micro-scale control.
Future work should prove the convergence of our aggregation algorithms and explore a wider variety of maps.  

\addtolength{\textheight}{-12cm}   % This command serves to balance the column lengths
                                  % on the last page of the document manually. It shortens
                                  % the textheight of the last page by a suitable amount.
                                  % This command does not take effect until the next page
                                  % so it should come on the page before the last. Make
                                  % sure that you do not shorten the textheight too much.

%%%%%%%%%%%%%%%%%%%%%%%%%%%%%%%%%%%%%%%%%%%%%%%%%%%%%%%%%%%%%%%%%%%%%%%%%%%%%%%%



%%%%%%%%%%%%%%%%%%%%%%%%%%%%%%%%%%%%%%%%%%%%%%%%%%%%%%%%%%%%%%%%%%%%%%%%%%%%%%%%



%%%%%%%%%%%%%%%%%%%%%%%%%%%%%%%%%%%%%%%%%%%%%%%%%%%%%%%%%%%%%%%%%%%%%%%%%%%%%%%%
%\section*{APPENDIX}
%
%
%
\section*{ACKNOWLEDGMENT}
%
%This work was supported by the National Science Foundation under Grant No.\ \href{http://nsf.gov/awardsearch/showAward?AWD_ID=1553063}{ [IIS-1553063]} and \href{http://nsf.gov/awardsearch/showAward?AWD_ID=1619278}{[IIS-1634726]}.% <-this % stops a space
%This work was supported by the National Science Foundation under Grant No.\  \href{http://nsf.gov/awardsearch/showAward?AWD_ID=1619278}{[IIS-1634726]}
%and \href{https://www.nsf.gov/awardsearch/showAward?AWD_ID=1634726}{[CMMI-1634726]}
This work was supported by the National Science Foundation (\href{http://nsf.gov/awardsearch/showAward?AWD_ID=1553063}{ [IIS-1553063]} and \href{http://nsf.gov/awardsearch/showAward?AWD_ID=1619278}{[IIS-1619278]} to A.T.B and
and [IIS-1734732] and [CMMI-1737682] to M.J.K.).

%This work was support by the National Science Foundation (IIS 1634726 to A.T.B. and IIS 1734732 and CMMI 1737682 to M.J.K.).
%	This work was supported by the National Science Foundation under Grant No.\ \href{http://nsf.gov/awardsearch/showAward?AWD_ID=1553063}{ [IIS-1553063]} and \href{http://nsf.gov/awardsearch/showAward?AWD_ID=1619278}{[IIS-1619278]}.% <-this % stops a space

%%%%%%%%%%%%%%%%%%%%%%%%%%%%%%%%%%%%%%%%%%%%%%%%%%%%%%%%%%%%%%%%%%%%%%%%%%%%%%%%




\bibliographystyle{IEEEtran}
\bibliography{IEEEabrv,ParticleSwarmRef}




\end{document}
